%Para este capítulo se usará la abreviatura "fund".
\chapter{Fundamentos teóricos}
\label{fund}
%Introducción por hacer (pensarlo bien).
\section{Planteamiento del problema}
\subsection{Preliminares. Funciones lineales}
\subsection{Forma estándar y formas canónicas}
\section{Programación lineal y conjuntos convexos}
\subsection{Conjuntos convexos. Definición y terminología}
\subsection{Puntos extremos}
\subsection{Direcciones extremas}
\section{Teoremas fundamentales}
\section{Algoritmo del Símplex}
\subsection{Fundamentos teóricos}
\subsection{Recapitulación}
\section{Ejercicios resueltos}
\begin{exerc}
	contenidos...
\end{exerc}
\begin{solu}
	contenidos...
\end{solu}