%Para este capítulo se usará la abreviatura "simp".
\chapter{Algoritmo del Símplex}
\label{simp}
Este es un capítulo dedicado a aclarar las lagunas que presenta el algoritmo \ref{fund_alg_simplex}, desarrollando ya un algoritmo completo que tenga en cuenta todas las contigencias habidas y por haber, el algoritmo del Símplex.
\section{Cambios de base}
Llegados a este punto nos preguntamos qué datos necesita el algoritmo \ref{fund_alg_simplex} para ejecutarse. Para obtener la respuesta basta con mirar con cuidado los teoremas en los que se basa, con lo que concluimos que son necesarios
\begin{itemize}
	\item Las componentes asociadas a la base (o componentes básicas) del punto extremo. Para calcular qué vector sale de la base al mejorar el punto extremo (teorema \ref{fund_teo_mejoraSimplex}) y para devolverlas cuando se detecta la optimalidad.
	\item El vector de costes reducidos asociado a la base del punto extremo para aplicar los tests de optimalidad y no acotación. Además de para decidir qué vector entra a la base para mejorar el punto extremo.
	\item Los vectores auxiliares $y_j$ para aplicar los tests de no acotación (y en su caso devolver la dirección en la que la función objetivo decrece indefinidamente) y para calcular qué vector sale de la base al mejorar el punto extremo.
	\item El valor de la función objetivo al ser evaluada en el punto extremo, para devolverlo cuando se halle la solución óptima.
\end{itemize}
Cuando cambiamos de punto extremo, el teorema de mejora nos da explícitamnte la nueva base asociada al nuevo punto. Este nuevo punto tendrá otras componentes asociadas a la base e inducirá nuevos vectores de costes reducidos y vectores axiliares que necesitan ser calculados para ejecutar la nueva iteración.

La forma inocente de calcular todas estas cosas pasa por invertir la base asociada al nuevo punto extremo, lo cual es pecado mortal, pues es un trabajo computacionalmente muy pesado (del orden de $\mc{O}(n^3)$).

En lugar de eso lo que haremos será calcular todos los nuevos valores necesarios a partir de los anteriores. A ver que esto es posible y cómo nos dedicamos en esta sección.
\subsection{Vectores auxiliares y componentes básicas}
En primer lugar nos planteamos la ecuación \eqref{fund_eq_problema} respecto de la base asociada al punto extremo original, $B$, y la nueva base $B'=(B\setminus a_l)\cup a_k$.
\begin{gather}
	x_B=\overline{x_B}-B^{-1}Nx_N\label{simp_eq_general1}\\
	x_{B'}=\overline{x_{B'}}-B'^{-1}N'x_{N'}\label{simp_eq_general2}
\end{gather}
\begin{obs}[Vectores auxiliares]
	Nótese que, por las propiedades del producto de matrices se tiene que
	\begin{gather*}
		B^{-1}N=B^{-1}(a_{m+1}\cdots a_n)=(B^{-1}a_{m+1}\cdots B^{-1}a_n)=(y_{m+1}\cdots y_n)\\
		B'^{-1}N'=(y'_{m+1}\cdots y'_{k-1}|y'_l|y'_{k+1}\cdots y'_n)\qedhere
	\end{gather*}
\end{obs}
Planteando las ecuaciones \eqref{simp_eq_general1} y \eqref{simp_eq_general2} componente a componente nos econtramos con
\begin{gather}
	x_s=\overline{x_s}-\sum_{\substack{j=m+1\\j\not=k}}^{n}y_{sj}x_j-y_{sk}x_k\label{simp_eq_componentes1}\\
	x_s=\overline{x_s'}-\sum_{\substack{j=m+1\\j\not=k}}^{n}y'_{sj}x_j-y'_{sl}x_l\label{simp_eq_componentes2}
\end{gather}
Una vez organizado nuestro espacio de trabajo como lo está ahora, vayamos por partes. Por un lado consideramos la ecuación \eqref{simp_eq_componentes1} para $s=l$. Despejando $x_k$ de esta ecuación obtenemos (¡compruébese!)
\begin{equation}
\label{simp_eq_xk}
	x_k=\frac{\overline{x_l}}{y_{lk}}-\sum_{\substack{j=m+1\\j\not=k}}^{n}\frac{y_{lj}}{y_{lk}}x_j-\frac{1}{y_{lk}}x_l
\end{equation}
Nótese que el despeje que hemos hecho es válido, ya que por el teorema \ref{fund_teo_mejoraSimplex} tenemos garantizado que $y_{lk}>0$. Ahora, si sustituimos el valor de $x_k$ dado por la ecuación \eqref{simp_eq_xk} en la ecuación \eqref{simp_eq_componentes1} obtenemos
\begin{equation*}
	x_s=\overline{x_s}-\frac{y_{sk}\overline{x_l}}{y_{lk}}+\sum_{\substack{j=m+1\\j\not=k}}^{n}\left(\frac{y_{sk}y_{lj}}{y_{lk}}-y_{sj}\right)x_j+\frac{y_{sk}}{y_{lk}}x_l
\end{equation*}
esta ecuación y la ecuación \eqref{simp_eq_componentes2} son dos ecuaciones lineales equivalentes. Por tanto, son la una múltiplo de la otra, no obtante, al tener $x_s$ el mismo coeficiente en ambas ecuaciones, estas deben ser iguales. De esto se desprende, comparando ambas ecuaciones
\begin{equation}
\label{simp_eq_aux1}
	\begin{array}{ccc}
		\displaystyle{\overline{x_s'}=\overline{x_s}-\frac{y_{sk}\overline{x_l}}{y_{lk}}}&\qquad
		\displaystyle{y_{sj}'=y_{sj}-\frac{y_{sk}y_{lj}}{y_{lk}}}&\qquad\displaystyle{y_{sl}'}=-\frac{y_{sk}}{y_{kl}}
	\end{array}
\end{equation}
para $s\in\{1,\dots,m\}\setminus\{l\}$ y con $j\in\{m+1,\dots,n\}\setminus\{k\}$. En el caso $s=l$ estas expresiones también son válidas (aunque no nos importa mucho).

Nos queda pues el trabajo de hallar extresiones para $\overline{x_k'}$, la coordenada $k$--ésima del nuevo punto extremo, e $y_{kj}'$ con $j\in\{m+1,\dots n\}\setminus\{k\}$ (las columnas de $N'$, excepto la columna $l$, que ya se quedó calculada) ya que $y_j'$ donde $j$ es un índice correspondiente a las columnas de $B'$ es $e_j$ (véase definición \ref{fund_defi_vectorAux}).

Cosideramos ahora la ecuación \eqref{simp_eq_componentes2} en el caso $s=k$. Dicha ecuación y \eqref{simp_eq_xk} son ecuaciones lineales equivalentes, de hecho iguales (siguiendo el razonamiento anterior). Por ende basta compararlas para obtener
\begin{equation}
\label{simp_eq_aux2}
	\begin{array}{ccc}
	\displaystyle{\overline{x_k'}=\frac{\overline{x_l}}{y_{lk}}} \qquad&\displaystyle{ y_{kj}'=\frac{y_{lj}}{y_{lk}}}\qquad&\displaystyle{y_{kl}'=\frac{1}{y_{lk}}}
	\end{array}
\end{equation}
\subsection{Costes reducidos y función objetivo}
Echando mano de la ecuación \eqref{fund_eq_funcionObj} respecto de las bases $B$ y $B'$ obtenemos
\begin{gather}
	c^tx=c_B^t\overline{x_B}+\sum_{\substack{j=m+1\\j\not=k}}^{n}\overline{c_j}x_j+\overline{c_k}x_k\label{simp_eq_obj1}\\
	c^tx=c_{B'}^t\overline{x_{B'}}+\sum_{\substack{j=m+1\\j\not=k}}^{n}\overline{c_j'}x_j+\overline{c_l'}x_l\label{simp_eq_obj2}
\end{gather}
Sustituyendo el valor de $x_k$ de \eqref{simp_eq_xk} en \eqref{simp_eq_obj1} obtenemos
\begin{equation*}
	c^tx=c_B^t\overline{x_B}+\overline{c_k}\frac{\overline{x_l}}{y_{lk}}+\sum_{\substack{j=m+1\\j\not=k}}^{n}\left(\overline{c_j}-\overline{c_k}\frac{y_{lj}}{y_{lk}}\right)x_j-\overline{c_k}\frac{1}{y_{lk}}x_l
\end{equation*}
tanto esta ecuación como la \eqref{simp_eq_obj2} son expresiones analíticas de la misma función lineal respecto de las mismas bases (que no tienen nada que ver con las bases asociadas a la matriz del problema). Esto quiere decir que ambas expresiones son iguales, luego comparándolas obtenemos
\begin{equation}
\label{simp_eq_reducidos}
	\begin{array}{ccc}
	\displaystyle{c_{B'}^t\overline{x_{B'}}=c_B^t\overline{x_B}+\overline{c_k}\frac{\overline{x_l}}{y_{lk}}}\qquad&\displaystyle{\overline{c_j'}=\overline{c_j}-\overline{c_k}\frac{y_{lj}}{y_{lk}}}\qquad&
	\displaystyle{\overline{c_l'}=-\overline{c_k}\frac{1}{y_{lk}}}\end{array}
\end{equation}
De esta forma, las ecuaciones \eqref{simp_eq_reducidos}, \eqref{simp_eq_aux1} y \eqref{simp_eq_aux2} nos proporcionan las fórmulas que andábamos buscando.
\section{Tabla del Símplex. Pivotajes}
En esta sección introducimos la llamada \tbi[tabla!del símplex]{tabla del símplex}, que no es más que una forma elegante de implementar las fórmulas de cambio de base deducidas en la sección anterior. Además, evita su memorización.

Sea un problema de programación lineal en forma estándar del que suponemos conocido un punto extremo de su región factible, es decir, una base $B$.

Consideremos el sistema de ecuaciones lineales asociado a dicho problema y multipliquémoslo a ambos lados por $B^{-1}$.
\begin{multline}
	\label{simp_eq_tabla}
	Ax=b\sii (B|N)(x_B|x_N)^t=b\sii B^{-1}(B|N)(x_B|x_N)^t=B^{-1}b\sii\\
	\sii (B^{-1}B|B^{-1}N)(x_B|x_N)^t=\overline{x_B}\sii (I_m|y_{m+1}\cdots y_n)(x_B|x_N)^t=\overline{x_B}
\end{multline}
La igualdad final de la ecuación \eqref{simp_eq_tabla} puede representarse como una tabla de la siguiente manera
\begin{table}[H]
	\centering
	\begin{tabular}{c|cccccccccc|c}
		$B$      & $x_1$    & $\cdots$ & $x_l$    & $\cdots$ & $x_m$    & $x_{m+1}$   & $\cdots$ & $x_k$    & $\cdots$ & $x_n$    & $\overline{x_B}$ \\ \hline
		$x_1$    & $1$      & $\cdots$ & $0$      & $\cdots$ & $0$      & $y_{1,m+1}$ & $\cdots$ & $y_{1k}$ & $\cdots$ & $y_{1n}$ & $\overline{x_1}$ \\
		$\cdots$ & $\cdots$ &          & $\cdots$ &          & $\cdots$ & $\cdots$    &          & $\cdots$ &          & $\cdots$ & $\cdots$         \\
		$x_l$    & $0$      & $\cdots$ & $1$      & $\cdots$ & $0$      & $y_{l,m+1}$ & $\cdots$ & $y_{lk}$ & $\cdots$ & $y_{ln}$ & $\overline{x_l}$ \\
		$\cdots$ & $\cdots$ &          & $\cdots$ &          & $\cdots$ & $\cdots$    &          & $\cdots$ &          & $\cdots$ & $\cdots$         \\
		$x_m$    & $0$      & $\cdots$ & $0$      & $\cdots$ & $1$      & $y_{m,m+1}$ & $\cdots$ & $y_{mk}$ & $\cdots$ & $y_{mn}$ & $\overline{x_m}$ \\ \hline
	\end{tabular}
	\label{simp_tab_simplex1}
	\caption{Primera aproximación a la tabla del Símplex.}
\end{table}
La virtud de representar un problema de programación lineal de esta forma es que los cambios de base son ``automáticos'' en cierto sentido. Supongamos que queremos cambiar de punto extremo de forma que la nueva base es $B'=(B\setminus a_l)\cup a_k$. Siguiendo las fórmulas de la sección anterior, la tabla asociada a la nueva base sería
\begin{table}[H]
	\centering
	\scalebox{0.9}[0.9]{
	\begin{tabular}{c|cccccccccc|c}
		$B'$     & $x_1$    & $\vdots$ & $x_l$                    & $\vdots$ & $x_m$    & $x_{m+1}$                                  & $\vdots$ & $x_k$    & $\vdots$ & $x_n$                                & $\overline{x_{B'}}$                                  \\ \hline
		$x_1$    & $1$      & $\vdots$ & $-\frac{y_{1k}}{y_{lk}}$ & $\vdots$ & $0$      & $y_{1,m+1}-y_{1k}\frac{y_{l,m+1}}{y_{lk}}$ & $\vdots$ & $0$      & $\vdots$ & $y_{1n}-y_{1k}\frac{y_{ln}}{y_{lk}}$ & $\overline{x_1}-y_{1k}\frac{\overline{x_l}}{y_{lk}}$ \\
		$\vdots$ & $\vdots$ &          & $\vdots$                 &          & $\vdots$ & $\vdots$                                   &          & $\vdots$ &          & $\vdots$                             & $\vdots$                                             \\
		$x_k$    & $0$      & $\vdots$ & $\frac{1}{y_{lk}}$       & $\vdots$ & $0$      & $\frac{y_{l,m+1}}{y_{lk}}$                 & $\vdots$ & $1$      & $\vdots$ & $\frac{y_{ln}}{y_{lk}}$              & $\frac{\overline{x_l}}{y_{lk}}$                      \\
		$\vdots$ & $\vdots$ &          & $\vdots$                 &          & $\vdots$ & $\vdots$                                   &          & $\vdots$ &          & $\vdots$                             & $\vdots$                                             \\
		$x_m$    & $0$      & $\vdots$ & $-\frac{y_{mk}}{y_{lk}}$ & $\vdots$ & $1$      & $y_{m,m+1}-y_{mk}\frac{y_{l,m+1}}{y_{lk}}$ & $\vdots$ & $0$      & $\vdots$ & $y_{mn}-y_{mk}\frac{y_{ln}}{y_{kl}}$ & $\overline{x_m}-y_{mk}\frac{\overline{x_l}}{y_{lk}}$ \\ \hline
	\end{tabular}
	}
	\caption{Tabla del problema respecto de la base $B'$.}
\end{table}
Si nos fijamos, la nueva tabla puede verse como el resultado de aplicar el algoritmo de Gauss--Jordan en la columna $k$--ésima, es decir, considerando la tabla como una matriz, usar las transformaciones elementales de matrices para convertir el elemento $y_{lk}$ en un $1$ y ``hacer ceros'' en el resto de la columna. Esto es maravilloso porque puede implementarse en ordenador de forma casi trivial.

La mala noticia es que la tabla \ref{simp_tab_simplex1} no tiene todos los datos necesarios para que el algoritmo se ejecute. En concreto, faltan los costes reducidos y el valor de la función objetivo.

La gran noticia sin embargo (y prueba irrefutable de que los dioses son benévolos de cuando en cuando) es que podemos completar la tabla \ref{simp_tab_simplex1} de manera que lleve todos los datos necesarios y además la implementacón de los cambios de base no varíe en absoluto. En efecto, si consideramos la tabla
\begin{table}[H]
	\centering
	\begin{tabular}{c|cccccccccc|c}
		$B$                                 & $x_1$                   & $\cdots$                     & $x_l$                   & $\cdots$                     & $x_m$                   & $x_{m+1}$                                & $\cdots$                     & $x_k$                                & $\cdots$                    & $x_n$                                 & $\overline{x_B}$                               \\ \hline
		$x_1$                               & $1$                     & $\cdots$                     & $0$                     & $\cdots$                     & $0$                     & $y_{1,m+1}$                              & $\cdots$                     & $y_{1k}$                             & $\cdots$                    & $y_{1n}$                              & $\overline{x_1}$                               \\
		$\cdots$                            & $\cdots$                &                              & $\cdots$                &                              & $\cdots$                & $\cdots$                                 &                              & $\cdots$                             &                             & $\cdots$                              & $\cdots$                                       \\
		$x_l$                               & $0$                     & $\cdots$                     & $1$                     & $\cdots$                     & $0$                     & $y_{l,m+1}$                              & $\cdots$                     & $y_{lk}$                             & $\cdots$                    & $y_{ln}$                              & $\overline{x_l}$                               \\
		$\cdots$                            & $\cdots$                &                              & $\cdots$                &                              & $\cdots$                & $\cdots$                                 &                              & $\cdots$                             &                             & $\cdots$                              & $\cdots$                                       \\
		$x_m$                               & $0$                     & $\cdots$                     & $0$                     & $\cdots$                     & $1$                     & $y_{m,m+1}$                              & $\cdots$                     & $y_{mk}$                             & $\cdots$                    & $y_{mn}$                              & $\overline{x_m}$                               \\ \hline
		\multicolumn{1}{c|}{$\overline{c}$} & \multicolumn{1}{c}{$0$} & \multicolumn{1}{c}{$\cdots$} & \multicolumn{1}{c}{$0$} & \multicolumn{1}{c}{$\cdots$} & \multicolumn{1}{c}{$0$} & \multicolumn{1}{c}{$\overline{c_{m+1}}$} & \multicolumn{1}{c}{$\cdots$} & \multicolumn{1}{c}{$\overline{c_k}$} & \multicolumn{1}{c}{$\cdots$} & \multicolumn{1}{c|}{$\overline{c_n}$} & \multicolumn{1}{c}{$-(c_{B}^t\overline{x_B})$}
	\end{tabular}
	\caption{Tabla del Símplex.}
\end{table}
la tabla asociada a la base $B'$ será
\begin{table}[H]
	\centering
	\scalebox{0.9}[0.9]{
	\begin{tabular}{c|cccccccccc|c}
		$B'$           & $x_1$    & $\vdots$ & $x_l$                             & $\vdots$ & $x_m$    & $x_{m+1}$                                                   & $\vdots$ & $x_k$    & $\vdots$ & $x_n$                                                & $\overline{x_{B'}}$                                                      \\ \hline
		$x_1$          & $1$      & $\vdots$ & $-\frac{y_{1k}}{y_{lk}}$          & $\vdots$ & $0$      & $y_{1,m+1}-y_{1k}\frac{y_{l,m+1}}{y_{lk}}$                  & $\vdots$ & $0$      & $\vdots$ & $y_{1n}-y_{1k}\frac{y_{ln}}{y_{lk}}$                 & $\overline{x_1}-y_{1k}\frac{\overline{x_l}}{y_{lk}}$                     \\
		$\vdots$       & $\vdots$ &          & $\vdots$                          &          & $\vdots$ & $\vdots$                                                    &          & $\vdots$ &          & $\vdots$                                             & $\vdots$                                                                 \\
		$x_k$          & $0$      & $\vdots$ & $\frac{1}{y_{lk}}$                & $\vdots$ & $0$      & $\frac{y_{l,m+1}}{y_{lk}}$                                  & $\vdots$ & $1$      & $\vdots$ & $\frac{y_{ln}}{y_{lk}}$                              & $\frac{\overline{x_l}}{y_{lk}}$                                          \\
		$\vdots$       & $\vdots$ &          & $\vdots$                          &          & $\vdots$ & $\vdots$                                                    &          & $\vdots$ &          & $\vdots$                                             & $\vdots$                                                                 \\
		$x_m$          & $0$      & $\vdots$ & $-\frac{y_{mk}}{y_{lk}}$          & $\vdots$ & $1$      & $y_{m,m+1}-y_{mk}\frac{y_{l,m+1}}{y_{lk}}$                  & $\vdots$ & $0$      & $\vdots$ & $y_{mn}-y_{mk}\frac{y_{ln}}{y_{kl}}$                 & $\overline{x_m}-y_{mk}\frac{\overline{x_l}}{y_{lk}}$                     \\ \hline
		$\overline{c'}$ & $0$      & $\vdots$ & $-\overline{c_k}\frac{1}{y_{lk}}$ & $\vdots$ & $0$      & $\overline{c_{m+1}}-\overline{c_k}\frac{y_{l,m+1}}{y_{lk}}$ & $\vdots$ & $0$      & $\vdots$ & $\overline{c_n}-\overline{c_k}\frac{y_{ln}}{y_{lk}}$ & $-c_{B'}^t\overline{x_{B'}}-\overline{c_k}\frac{\overline{x_l}}{y_{lk}}$
	\end{tabular}}
	\caption{Tabla del problema respecto de la base $B'$.}
\end{table}
Como se observa, la tabla del Símplex tiene en su última fila el vector de costes reducidos y el \tb{opuesto} al valor que toma la función objetivo en el punto extremo asociado a la base que corresponda. La razón para guardar el opuesto y no el valor a secas es que si guardamos el opuesto podemos seguir haciendo los cambios de base como si aplicaramos parcialmente el algoritmo de Gauss-Jordan, en caso contrario no podríamos hacerlo.

\begin{obs}[Notación]
	Es habitual encontrarse con que la casilla correspondiente a la función objetivo en la tabla del Símplex contiene la expresión $z-(c_B^t\overline{x_B})$. Esto no supone ninguna variación en absolutamente nada, es símplemente notación, añadir una $z$ al principio, sin más.
\end{obs}

A la operación de cambiar de base una tabla se la denmina \tbi{pivotaje} y al elemento $y_{lk}$ se le denomina \tbi{pivote}.
\section{Punto extremo inicial}
Aunque el trabajo duro ya está hecho, todavía quedan preguntas por resolver, por ejemplo, cómo obtener el punto extremo inicial (si lo hay, ya que el problema podría ser infactible) y todos los datos necesarios asociados a este de forma eficiente.
\subsection{Método de las dos fases}
Encontrar bases de matrices es un trabajo compuracionalmente costoso, en el caso peor, que no haya base alguna, es del orden de $\mc{O}\binom{n}{m}$ (eso si las comprobaciones intermedias tuvieran coste constante, que no lo tienen). Razón por la cual necesitamos algún tipo de sustitutivo a este método de fuerza bruta.

Lo que exponemos a continuación es conocido como \tbi[método!de las dos fases]{método de las dos fases} y únicamente es aplicable a problemas en forma estándar con $b\geq 0$. Esto no supone ninguna pérdida de generalidad ya que siempre podemos multiplicar la restricción correspondiente a un $b_i<0$ por $-1$, obteniendo una restricción equivalente y, por tanto, un problema equivalente.

A un problema en las hipótesis anteriormente expuestas que tenga a $A$ por matriz de coeficientes de las restricciones, se le asocia otro problema al que llamaremos \tbi[problema!artificial]{problema artificial}. Dicho problema es el siguiente
\begin{equation*}
	\begin{array}{c}
		\min w(x|x_a)^t:=\sum_{i=1}^{m}x_i^a\\
		\text{Sujeto a:}\qquad (A|I_m)(x|x_a)^t=b,\qquad x\geq 0,\ x_a\geq 0
	\end{array}
\end{equation*}
es decir, el problema resultante de cosiderar las restricciones
\begin{equation*}
	a_{i1}x_1+\dots+a_{in}x_n+x_i^a=b_i
\end{equation*}
para $i\in\{1,\dots,m\}$, cuya matriz asociada es $(A|I_m)$. A las nuevas variables $x_i^a$ se las conoce como \tbi[variables!artificiales]{variables artificiales}. Además, la función objetivo se sustituye por $w\equiv \sum_{i=1}^{m}x_i^a$.

Entre las buenas propiedades de este nuevo problema es que tiene una base trivial, la identidad, situada convenientemente en las últimas columnas (lo cual el ordenador agradece).

Pero la cosa no queda aquí, ya que el problema artificial no es un problema no acotado. En efecto, por las restricciones de no negatividad, la función objetivo está acotada inferiormente por $0$. Además, el problema artificial es factible, bastanto considerar el vector $(0|b)^t\in\R^{n+m}$ como solución factible. Esta es la razón por la que exigimos $b\geq 0$.

Con esto, por el teorema fundamental de la programación lineal (teorema \ref{fund_teo_fund}), sabemos que el problema artificial tiene solución óptima. Por si esto fuera poco, obtenemos casi gratis un test de factibilidad para el problema original.
\begin{lem}[Test de factibilidad]
	\label{simp_lem_testFacti}
	El problema original es factible si y solo si el problema artificial posee una solución óptima $(\overline{x}|\overline{x_a})^t$ con $\overline{x_a}=0$.
\end{lem}
\begin{proof}
	Demostraremos la implicación a la derecha, para descubrir después que si miramos la demostración con amor, y del revés, la otra implicación también está demostrada.
	
	Si el problema original es factible, entonces habrá un $x\geq 0$ tal que $Ax=b$. En tal caso es claro que el vector $(x|0)^t$ es solución factible del problema artificial (¡compruébese!). Como resulta que $w(x|0)^t=0$, por la acotación inferior de $w$ tenemos que $(x|0)^t$ es solución óptima del problema artificial con $\overline{x_a}=0$.
\end{proof}
\begin{obs}[Sutileza]
	Si el problema artificial posee una solución óptima $x^*$ con $\overline{x_a}=0$, todas sus soluciones óptimas $x'$ tendrán $\overline{x_a}=0$, ya que, en caso contrario $w(x^*)<w(x')$, lo cual es absurdo. 
\end{obs}
Dicho esto, es claro que, dado un problema de programación lineal dentro de nuestras hipótesis, podemos considerar el problema artificial asociado y resolverlo mediante el algoritmo del Símplex que ya tenemos desarrollado, siendo el rellenado de la tabla inicial (y la determinación del primer punto extremo) trivial, ya que $B=I_m$, por tanto la tabla del Símplex es la correspondiente a la matriz $(A|I_m|b)$ añadiendo como última fila el vector de costes reducidos y el opuesto del valor de la función objetivo.

Cabe destacar que el cálculo del vector de costes reducidos es bastante trivial computacionalmente ya que no hay que invertir ninguna base, símplemente $\overline{w_j}=w_j-w_B^ta_j$.

Una vez hallada la solución óptima al problema artificial, podremos usar el lema \ref{simp_lem_testFacti} para decidir sobre la factibilidad o no del problema original. En caso de que este resulte factible, la base asociada a la solución óptima encontrada será base del problema original (ya que todas las columnas de la base están asociadas a columnas de la matriz original). Con esto, ya podríamos aplicar el algoritmo del Símplex para resolver el problema original.

No obstante, cabría preguntarse si es computacionalmente costoso rellenar la tabla del simplex del problema original respecto de la base $B$ dada (es decir, ¿hay que invertir $B$?). La respuesta es no, ya que necesitamos calcular la matriz
\begin{equation*}
	B^{-1}(A|b)=(B^{-1}a_1\cdots B^{-1}a_n|B^{-1}b)
\end{equation*}
y la matriz asociada a la tabla asociada a la solución óptima del problema artificial es
\begin{equation*}
	B^{-1}(A|I_m|b)=(B^{-1}a_1\cdots B^{-1}a_n|B^{-1}e_1\cdots B^{-1}e_m|B^{-1}b)=(B^{-1}A|B^{-1}|B^{-1}b)
\end{equation*}
luego basta con tomar la misma tabla suprimiendo las columnas asociadas a las variables artificiales.

En cuanto al cálculo de los costes reducidos, aunque ya tenemos invertida la matriz $B$ y podríamos hacerlo ``a capón'', no es recomendable, pues multiplicar matrices es computacionalemente costoso.

En lugar de eso, lo que podemos hacer es añadir una fila más a la tabla del Símplex del problema artificial, donde ir calculando (mediante los pivotajes) los costes reducidos asociados a la función objetivo $z:=c^tx$ del problema original (nótese que no hay ningún problema).

De esta forma, basta con consultar la tabla asociada a $B$ del problema artificial para rellenar la tabla inicial del problema original, que ya puede ser resuelto usando el algoritmo del Símplex conocido. Organizemos toda esta literatura.
\begin{algorithm}[H]
	\begin{algorithmic}[1]
		\STATE\COMMENT{Primera fase}
			\STATE Construir el problema artificial
			\STATE Resolver el problema artificial (algoritmo \ref{fund_alg_simplex}) repedidas veces.
			\STATE Aplicar el test de factibilidad (lema \ref{simp_lem_testFacti})
			\IF{es factible}
				\STATE\COMMENT{Segunda fase}
				\STATE Rellenar la primera tabla con los trucos vistos.
				\STATE Resolver el problema original (algoritmo \ref{fund_alg_simplex}) repedidas veces.
			\ELSE
				\RETURN Infactible
			\ENDIF
	\end{algorithmic}
	\caption{Algoritmo de las dos fases.}\label{simp_alg_dosFases}
\end{algorithm}
\subsection{Método de las penalizaciones}
\section{Prevención de bucles}
\subsection{Regla lexicográfica}
\subsection{Regla de Bland}