\chapter*{Prefacio}
Estas notas son una transcripción (libremente adaptada) de las clases de la asignatura ``\ti{Investigación Operativa}'', impartidas por María Inés Sobrón Fernández en el curso 2016--2017 a los cursos de cuarto y tercero de los dobles grados de  Matemáticas -- Física e Ingeniería Informática -- Matemáticas (respectivamente) en la facultad de Ciencias Matemáticas de la Universidad Complutense de Madrid (UCM).

Cualquier aportación o sugerencia de mejora es siempre bienvenida.
\subsection*{Requisitos previos}
Para comprender estas notas en su totalidad es necesario tener soltura a la hora de trabajar con matrices, y, en general, haber entendido bien el álgebra lineal. También es bastante recomendable recordar algunos aspectos del cálculo diferencial en varias variables y teoría de grafos, no obstante, el texto es bastante autocontenido en ese aspecto.
\subsection*{Qué contiene este texto}
Este texto contiene \tb{todos} los contenidos teóricos de la asignatura (según se impartió en el curso $2016$--$2017$) sin los cuales es totalmente imposible comprender los prácticos. 

Este texto \tb{no} contiene apenas ejemplos ni ejercicios prácticos, al momento de la elaboración de este texto estas carencias se consideraron menores debido a la gran cantidad de ejemplos disponibles en comparación con la escasez de teoría rigurosa.

Estoy totalmente seguro de que este texto contiene numerosas erratas, motivo por el cual se agradece al lector cuidadoso que, de una manera u otra me las haga llegar.
\subsection*{Agradecimientos}
La existencia de estas notas es debida a la amabilidad de Clara Rodríguez Núñez, quien me cedió sus apuntes tomados durante el curso, en los cuales se basa el núcleo de este texto.
\subsection*{Licencia}
Esta obra está sujeta a la licencia Reconocimiento-NoComercial-CompartirIgual 4.0 Internacional de Creative Commons. Para ver una copia de esta licencia, visite \url{http://creativecommons.org/licenses/by-nc-sa/4.0/}.
\begin{figure}[h]
	\centering
	\includegraphics[scale=1]{img/licencia}
\end{figure}

El código fuente de este documento es de libre acceso y se encuentra alojado en \url{https://github.com/alvarogatenorio/Investigacion-Operativa} paro uso y disfrute de todo el que quiera, siempre que se respeten los términos de la licencia.