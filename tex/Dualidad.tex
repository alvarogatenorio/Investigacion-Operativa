%Para este capítulo se usará la abreviatura "dual".
\chapter{Dualidad}
\label{dual}
En este capítulo introduciremos los llamados ``problemas duales'', estudiando sus propiedades y relaciones con sus respectivos ``problemas primales''. Sacaremos jugo a estas propiedades cual Jíbaro a la cabeza de un enemigo.
\section{Formulación canónica del problema dual}
Sea un problema $P$ de programación lineal en forma canónica de maximización. Sintéticamente
\begin{equation*}
	\begin{array}{c}
		\max c^tx\\
		\text{Sujeto a:}\qquad Ax\leq b,\qquad x\geq 0
	\end{array}
\end{equation*}
se define el \tbi[problema!dual]{problema dual} $D$ asociado a $P$ como
\begin{equation*}
	\begin{array}{c}
		\min b^tu\\
		\text{Sujeto a:}\qquad A^tu\geq c,\qquad u\geq 0
	\end{array}
\end{equation*}
usualmente nos referiremos al problema $P$ como el \tbi[problema!primal]{problema primal}.
\begin{obs}[Involutividad]
	La primera cosa que salta a la vista es que la dualidad es involutiva, es decir, el problema dual asociado al problema dual es el problema primal (esto lo veremos en el ejemplo \ref{dual_exa_min}), que es de vital importancia.
\end{obs}
Aunque esta definición de problema dual parezca estricta, por solo poder aplicarse a los problemas en forma canónica de maximización, en realidad no lo es tanto, ya que podemos pasar de una forma canónica a otra. Veamos, por ejemplo, cuál es el problema dual asociado a un problema en forma canónica de minimización.
\begin{exa}[Minimización]
	\label{dual_exa_min}
	Teniendo en cuenta que $\min f = -\max -f$ (se deja al lector la comprobación) tenemos que, dado el problema en forma canónica de minimización
	\begin{equation*}
		\begin{array}{c}
			\min c^tx\\
			\text{Sujeto a:}\qquad Ax\geq b,\qquad x\geq 0
		\end{array}
	\end{equation*}
	su transformación a forma canónica de maximización es
	\begin{equation*}
		\begin{array}{c}
			-\max -c^tx\\
			\text{Sujeto a:}\qquad -Ax\leq -b,\qquad x\geq 0
		\end{array}
	\end{equation*}
	donde lo único que se ha hecho es multiplicar por $-1$ todas las restricciones así como la función objetivo. En esta situación el problema dual ya está bien definido, y es
	\begin{equation*}
		\begin{array}{c}
			-\min -b^tu\\
			\text{Sujeto a:}\qquad -A^tu\geq -c,\qquad u\geq 0
		\end{array}
	\end{equation*}
	planteando esto de nuevo en forma canónica de maximización obtenemos
	\begin{equation*}
		\begin{array}{c}
			\max b^tu\\
			\text{Sujeto a:}\qquad A^tu\leq c,\qquad u\geq 0
		\end{array}
	\end{equation*}
	con lo que ya tenemos el problema dual asociado al problema original.
\end{exa}
Nótese que no está muy claro cómo definir el problema dual asociado a un problema en forma estándar (mucho menos en otras formas más exóticas). A esta cuestión nos decicaremos en secciones posteriores, justo después de ver por qué merece la pena plantearse estos problemas.
\section{Relaciones de dualidad}
En esta sección veremos cómo se relacionan un problema y su dual en términos de sus soluciones, viendo así parte de la utilidad de estudiar la dualidad. Comencemos con el siguiente resultado elemental.
\begin{lem}[Cotas]
	\label{dual_lem_cotas}
	Toda solución factible del problema dual proporciona una cota superior del valor óptimo de la función objetivo del problema primal, y viceversa. Es decir
	\begin{equation*}
		c^tx\leq b^tu
	\end{equation*}
\end{lem}
\begin{proof}
	Sea $x\in\R^n$ una solución factible del problema primal. Asimismo consideremos $u\in\R^m$ una solución factible del problema dual.
	
	Tenemos que $c^tx=\sum_{j=1}^nc_jx_j$. Teniendo en cuenta la definición del problema dual ($A^tu\geq c$) tenemos que $c^tx\leq \sum_{j=1}^n(\sum_{i=1}^{m}a_{ij}u_i)x_j$. Intercambiando los sumatorios obtenemos la expresión $c^tx\leq\sum_{i=1}^{m}(\sum_{j=1}^{n}a_{ij}x_j)u_i$. Teniendo en cuenta la definición del problema primal $(Ax\leq b)$ concluimos que $c^tx\leq \sum_{i=1}^{m}bu_i=b^tu$. Como queríamos demostrar.
\end{proof}
\begin{cor}[Dualidad débil]
	\label{dual_cor_dualidadDebil}
	Si dos soluciones factibles $x$ y $u$, del problema primal y dual respectivamente, verifican que $c^tx=b^tu$, entonces, tanto $x$ como $u$ son soluciones óptimas de sus respectivos problemas.
\end{cor}
El siguiente resultado es conocido en la literatura con el nombre de ``teorema de dualidad'' o ``teorema de dualidad fuerte''. Es el recíproco del corolario \ref{dual_cor_dualidadDebil}. Veámoslo.
\begin{theo}[Dualidad fuerte]
	\label{dual_teo_dualidadFuerte}
	Si el problema primal tiene solución óptima $x$, entonces el problema dual tiene solución óptima $u$, y se verifica que $c^tx=b^tu$.
\end{theo}
\begin{proof}
	Podemos suponer si pérdida de generalidad que la solución óptima $x$ se ha obtenido mediante el algoritmo del Símplex. En tal, caso llamaremos $B$ a su base asociada. Asimismo consideremos el vector fila $\lambda^t:=c_B^tB^{-1}$.
	
	Como $x$ es solución óptima de un problema de maximización encontrada mediante el algoritmo del Símplex se deberá verificar que todos los costes reducidos son no positivos, es decir, $\overline{c_j}\leq 0$.
	
	Nótese que para que se pueda ejecutar el algoritmo del Símplex sobre el problema primal se deberán añadir $m$ variables de holgura (una por restricción), de modo que la matriz del problema pasaría a ser $(A|I_m)$.
	
	Estudiemos con un poco de cariño los costes reducidos.
	\begin{equation*}
		\overline{c_j}=c_j-c_B^tB^{-1}a_j=c_j-\lambda^ta_j\leq 0\sii \lambda^ta_j\geq c_j\sii \sum_{i=1}^{m}a_{ij}\lambda_i\geq c_j
	\end{equation*}
	La última equivalencia nos dice que el vector $\lambda\in\R^m$ cumple con las restricciones del problema dual. Para asegurarse de que es solución factible de este basta con comprobar la no negatividad, para ello consideramos los costes reducidos asociados a las variables de holgura ($\overline{c_j}$ con $j\in\{n+1,\dots,n+m\}$), que se caracterizan por el hecho de que $c_{n+i}=0$ y $a_{n+i}=e_i$ con $i\in\{1,\dots,m\}$.
	\begin{equation*}
		\overline{c_{n+i}}=-\lambda^te_i=-\lambda_i\leq 0\sii\lambda_i\geq 0 
	\end{equation*}
	con lo que se concluye que $\lambda$ es solución factible dual. Además
	\begin{equation*}
		c^tx=\sum_{i=1}^{n}c_ix_i=c_B^t\overline{x_B}=c_B^tB^{-1}b=\lambda^tb=b^t\lambda
	\end{equation*}
	por el coralario \ref{dual_cor_dualidadDebil} tenemos que $\lambda$ es solución óptima del problema dual.
\end{proof}
\begin{obs}[Construcción]
	Nótese que a partir de la solución óptima del primal obtenida por el algoritmo del Símplex obtenemos automáticamente la solución óptima del dual, en concreto, si la solución óptima del primal está asociada a la base $B$, la solución óptima del dual es $(c_B^tB^{-1})^t=-(\overline{c_{n+1}},\dots,\overline{c_{n+m}})$. Luego podemos decir que la solución del problema dual está literalmente en la tabla del Símplex del primal.
\end{obs}
Antes de continuar es necesario hacer un pequeño inciso.
\begin{obs}[Teorema dual]
	\label{dual_obs_dual}
	Existe un ``teorema dual'' al teorema de dualidad fuerte \ref{dual_teo_dualidadFuerte}, cuyo enunciado es el mismo, pero permutando las palabras ``primal'' y ``dual''. La demostración es totalmente análoga y se deja al lector.
	
	Esto viene a significar que si alguno de los dos problemas tiene solución óptima, el otro también la tendrá.
\end{obs}

Otro resultado bastante fuerte que relaciona las soluciones de los problemas primal y dual es el llamado ``teorema de la holgura complementaria'', que da condiciones necesarias y suficientes para que dos soluciones (una del primal y otra del dual) sean óptimas simultáneamente.
\begin{theo}[Teorema de la holgura complementaria]
	\label{dual_teo_holgura}
	Sean $x$ e $u$ soluciones factibles de primal y dual respectivamente. Se verifica que $x$ e $u$ son soluciones óptimas de sus respectivos problemas si y solo si se cumplen
	\begin{enumerate}
		\item Para cada restricción del problema primal (suponemos la $i$--ésima con $i\in\{1,\dots,m\}$) se verifica que $\sum_{j=1}^{n}a_{ij}x_j=b_i$ o bien $u_i=0$.
		\item Para cada restricción del problema dual (suponemos la $j$--ésima con $j\in\{1,\dots,n\}$) se verifica que $\sum_{i=1}^{m}a_{ij}u_i=c_j$ o bien $x_j=0$.
	\end{enumerate}
\end{theo}
\begin{proof}
	Como por el lema \ref{dual_lem_cotas} sabemos que $c^tx\leq b^tu$, lo cual implica que tanto el problema primal como el problema dual tienen solución óptima (por no ser ni infactibles ni no acotados, véase teorema \ref{fund_teo_fund}).
	
	Dicho esto, el teorema \ref{dual_teo_dualidadFuerte} combinado con el corolario \ref{dual_cor_dualidadDebil} nos asegura que ser soluciones óptimas simultáneas es equivalente a que se verifique la igualdad $c^tx=b^tu$. Busquemos condiciones necesarias y suficientes para que esto se cumpla.
	
	Podemos resumir el contenido fundamental del lema \ref{dual_lem_cotas} en la siguiente cadena de desigualdades.
	\begin{equation*}
		c^tx=\sum_{j=1}^{n}c_jx_j\leq \sum_{j=1}^{n}\left(\sum_{i=1}^{m}a_{ij}u_i\right)x_j=\sum_{i=1}^{m}\left(\sum_{j=1}^{n}a_{ij}x_j\right)u_i\leq \sum_{i=1}^{m}b_iu_i=b^tu
	\end{equation*}
	Una condición suficiente para que se de la igualdad en la primera desigualdad de la cadena es que $c_jx_j=(\sum_{i=1}^{m}a_{ij}u_i)x_j$ para $j\in\{1,\dots,n\}$, lo cual se da si y solo si $c_j=\sum_{i=1}^{m}a_{ij}u_i$ o bien $x_j=0$. Encontramos una condición suciente análoga para la segunda desigualdad de la cadena.
	
	La condición que hemos impuesto también es necesaria, ya que $c_j\leq\sum_{i=1}^{m}a_{ij}u_i$ y $x_j\geq 0$ para todo $j\in\{1,\dots,n\}$, de este modo es fácil ver que si se diera la desigualdad estricta para un $j$ en el cual $x_j>0$ se produciría una desigualdad estricta también en el sumatorio. Análogamente se hace con la otra desigualdad.
\end{proof}
Aunque este último teorema pueda parece más inútil que un cubo de tela, no es así, pues, como vemos a continuación, sirve para comprobar si una solución (no necesariamente punto extremo) es óptima o no sin necesidad de aplicar el algoritmo del Símplex. Veamos este resultado en forma de una observación y un ejemplo.
\begin{obs}[Test de optimalidad]
	Una solución factible $x$ del problema primal es óptima si y solo si existe un $u\in\R^m$ que verifica las siguientes condiciones
	\begin{enumerate}
		\item Si $\sum_{j=1}^{n}a_{ij}x_j<b_i$ entonces $u_i=0$ y si $x_j>0$ entonces $\sum_{i=1}^{m}a_{ij}u_i=c_j$.
		\item Se verifica que $\sum_{i=1}^{m}a_{ij}u_i\geq c_j$ y $u_i\geq 0$ para todo $j\in\{1,\dots,m\}$.
	\end{enumerate}
	La corrección de este ``test'' se deduce trivialmente de los teoremas \ref{dual_teo_dualidadFuerte} y \ref{dual_teo_holgura} junto con el corolario \ref{dual_cor_dualidadDebil}. Se recomienda encarecidamente al lector poner a prueba esta observación en la práctica.
\end{obs}
Para finalizar la sección presentemos una tabla resumen que recopila casi todo lo aprendido.
\begin{table}[H]
	\centering
	\begin{tabular}{c|c|c|c|}
		\multicolumn{1}{l|}{} & Infactible & Sol. óptima & Sol. no acotada \\ \hline
		Infactible & \textbf{Posible} & \cellcolor[HTML]{C0C0C0}{\color[HTML]{FFCCC9} \textbf{}} & \textbf{Posible} \\ \hline
		Sol. óptima & \cellcolor[HTML]{C0C0C0}\textbf{} & \textbf{Posible} & \cellcolor[HTML]{C0C0C0}\textbf{} \\ \hline
		Sol. no acotada & \textbf{Posible} & \cellcolor[HTML]{C0C0C0}\textbf{} & \cellcolor[HTML]{C0C0C0}\textbf{} \\ \hline
	\end{tabular}
	\caption{Relaciones de dualidad.}
\end{table}
Como se puede observar, la tabla es totalmente simétrica, motivo por el cual no se ha especificado a en ningún letrero se nos referimos al problema dual o al problema primal.

La demostración de que la tabla es verídica es muy sencilla, siendo la demostración de la veracidad de la fila central el contenido de la observación \ref{dual_obs_dual}. Por su parte, la demostración correspondiente a la última fila se basa en el lema \ref{dual_lem_cotas}. Es muy sencilla y se deja al lector. En cuanto a la primera fila, se pueden encontrar ejemplos de los dos casos posibles, mientras que el caso otro caso es imposible por la observación \ref{dual_obs_dual}.
\section{Otras formulaciones}
Vamos a extender la noción ``problema dual asociado'' a problemas que no nevesariamente están presentados en forma canónica.
\subsection{Formulación estándar}
Dado un problema de programación lineal en forma estándar
\begin{equation*}
\begin{array}{c}
\min c^tx\\
\text{Sujeto a:}\qquad Ax= b,\qquad x\geq 0
\end{array}
\end{equation*}
vamos a ponerlo en forma canónica de minimización usando el siguiente truco. Consideramos el problema equivalente
\begin{equation*}
\begin{array}{c}
\min c^tx\\
\text{Sujeto a:}\qquad Ax\leq b\quad \&\quad Ax\geq b,\qquad x\geq 0
\end{array}
\end{equation*}
que a su vez transformamos en
\begin{equation*}
\begin{array}{c}
\min c^tx\\
\text{Sujeto a:}\qquad -Ax\geq -b\quad \&\quad Ax\geq b,\qquad x\geq 0
\end{array}
\end{equation*}
que ya está en forma canónica, por simplificar un poco las cosas explicitamos que la matriz de este problema es $(A^t|-A^t)^t$ y su vector de términos independientes $(b^t|-b^t)^t$. De esta forma ya podemos calcular el problema dual con la definición usual. Este es
\begin{equation*}
\begin{array}{c}
\max b^tx\\
\text{Sujeto a:}\qquad(A^t|-A^t)u\leq c\qquad u\geq 0
\end{array}
\end{equation*}
Si llamamos $u_1$ a las $m$ componentes de $u$ asociadas a la submatriz $A^t$, mientras llamamos $u_2$ a las componentes asociadas a la submatriz restante tenemos que el problema dual tiene por restricción $A^tu_1-A^tu_2\leq c$, que es equivalente a la restricción $A^t(u_1-u_2)\leq c$.

Llamando $w:=u_1-u_2$ tenemos que el problema dual es
\begin{equation*}
\begin{array}{c}
\max b^tx\\
\text{Sujeto a:}\qquad A^tw\leq c
\end{array}
\end{equation*}
Nótese que como $u_1,u_2\geq 0$, $w$ no tiene ninguna restricción en cuanto al signo.

Como la involutividad se sigue cumpliendo, ya tenemos una forma de ``dualizar'' problemas de maximización con variables no restringidas. Si uno no quiere gastar memoria en esto, basta con reproducir el procedimiento realizado y leerlo de abajo a arriba.

Un buen ejercicio consistiría en repetir el apartado entero cambiando el problema inicial a dualizar por un problema en forma estándar pero de maximización.
\subsection{Formulación general}
En este apartado vamos a ver cómo dualizar cualquier problema de programación lineal (independientemente de su formulación), para lo cual seguiremos una estrategia totalmente análoga a la del apartado anterior.

Uno de los problemas más generales que se nos puede plantear es el que tiene restricciones de todo tipo, es decir
 \begin{equation*}
 \begin{array}{c}
 \max c^tx\\
 \text{Sujeto a:}\qquad A_1x\leq b\quad \&\quad A_2x=b\quad\&\quad A_3x\geq b,\qquad x\geq 0
 \end{array}
 \end{equation*}
 Si lo transformamos en un problema en forma canónica obtenemos (¡compruébese!)
 \begin{equation*}
 \begin{array}{c}
 \max c^tx\\
 \text{Sujeto a:}\qquad (A_1^t|A_2^t|-A_2^t|-A_3^t)^tx\leq (b_1^t|b_2^t|b_3^t|b_4^t)^t\qquad x\geq 0
 \end{array}
 \end{equation*}
 por tanto su dual será (háganse las cuentas)
  \begin{equation*}
  \begin{array}{c}
  \min b^tx\\
  \text{Sujeto a:}\qquad (A_1^t|A_2^t|-A_2^t|-A_3^t)(u_1^t|u_2^t|u_3^t|u_4^t)^t\geq c\qquad u\geq 0
  \end{array}
  \end{equation*}
  Por ende la restricción del problema dual es equivalente a $A_1^tu_1+A_2^t(u_2-u_3)+A_3^t(-u_4)\geq c$. Llamando $w_1:=u_2-u_3$ y $w_2:=-u_4$ obtenemos que el problema dual tiene a $A_1^tu_1+A_2^tw_1+A_3^tw_2$ por restricción, donde $w_1$ son variables no restrigidas y $w_2$ son variables no positivas.
  
  Podemos realizar el mismo proceso con un problema de minimización, y en general con cualquier problema, ya que las anomalías de tener variables no restringidas o no positivas se trataron en el capítulo \ref{fund}, de modo que podemos transformar cualquier problema en uno del tipo tratado en este apartado o su análogo de minimización.
\section{Algoritmo dual}
Resolver problemas de programación lineal en forma canónica de minimización con el algoritmo de Símplex provoca, a simple vista, un desperdicio de memoria que da pena verlo.
\subsection{Fundamentación teórica}
\subsection{Método de la restricción artificial}