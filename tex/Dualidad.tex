%Para este capítulo se usará la abreviatura "dual".
\chapter{Dualidad}
\label{dual}
\section{Formulación del problema dual}
Sea un problema $P$ de programación lineal en forma canónica de maximización. Sintéticamente
\begin{equation*}
	\begin{array}{c}
		\max c^tx\\
		\text{Sujeto a:}\qquad Ax\leq b,\qquad x\geq 0
	\end{array}
\end{equation*}
se define el \tbi[problema!dual]{problema dual} $D$ asociado a $P$ como
\begin{equation*}
	\begin{array}{c}
		\min b^tu\\
		\text{Sujeto a:}\qquad A^tu\geq b,\qquad u\geq 0
	\end{array}
\end{equation*}
usualmente nos referiremos al problema $P$ como el \tbi[problema!primal]{problema primal}.
\begin{obs}[Involutividad]
	La primera cosa que salta a la vista es que la dualidad es involutiva, es decir, el problema dual asociado al problema dual es el problema primal (¡compruébese!).
\end{obs}
Aunque esta definición de problema dual parezca estricta, por solo poder aplicarse a los problemas en forma canónica de maximización, en realidad no lo es tanto, ya que podemos pasar de una forma canónica a otra. Veamos, por ejemplo, cuál es el problema dual asociado a un problema en forma canónica de minimización.
\begin{exa}[Minimización]
	Teniendo en cuenta que $\min f = -\max -f$ (se deja al lector la comprobación) tenemos que, dado el problema en forma canónica de minimización
	\begin{equation*}
		\begin{array}{c}
			\min c^tx\\
			\text{Sujeto a:}\qquad Ax\geq b,\qquad x\geq 0
		\end{array}
	\end{equation*}
	su transformación a forma canónica de maximización es
	\begin{equation*}
		\begin{array}{c}
			-\max -c^tx\\
			\text{Sujeto a:}\qquad -Ax\leq -b,\qquad x\geq 0
		\end{array}
	\end{equation*}
	donde lo único que se ha hecho es multiplicar por $-1$ todas las restricciones así como la función objetivo. En esta situación el problema dual ya está bien definido, y es
	\begin{equation*}
		\begin{array}{c}
			-\min -b^tu\\
			\text{Sujeto a:}\qquad -A^tu\geq -c,\qquad u\geq 0
		\end{array}
	\end{equation*}
	planteando esto de nuevo en forma canónica de maximización obtenemos
	\begin{equation*}
		\begin{array}{c}
			\max b^tu\\
			\text{Sujeto a:}\qquad A^tu\leq c,\qquad u\geq 0
		\end{array}
	\end{equation*}
	con lo que ya tenemos el problema dual asociado al problema original.
\end{exa}
Nótese que no está muy claro cómo definir el problema dual asociado a un problema en forma estándar. A esta cuestión nos decicaremos en secciones posteriores, justo después de ver por qué merece la pena plantearse estos problemas.
\section{Relaciones de dualidad}
En esta sección veremos cómo se relacionan un problema y su sual en términos de sus soluciones, viendo así parte de la utilidad de estudiar la dualidad.